\begin{table}[h!]
\caption{Collision severity ordinal classification and OCTraffic dataset counts}
\label{tbl1}
\begin{tabular}{lrrrrrcrrc}
\toprule
\multicolumn{4}{c}{} & \multicolumn{3}{c}{Party Count\footnotemark[1]} & \multicolumn{3}{c}{Victim Count\footnotemark[1]} \\
Severity Level & Crashes & Parties & Victims & \textit{mean} & \textit{std} & \textit{range} & \textit{mean} & \textit{std} & \textit{range} \\
\midrule
Possible injury or pain & 109,114 & 242,354 & 185,617 & 2.437 & 0.838 & 1-12 & 1.677 & 1.030 & 1-16 \\
Minor or visible injury & 57,779 & 119,327 & 103,039 & 2.349 & 0.940 & 1-14 & 2.010 & 1.311 & 1-17 \\
Serious or severe injury & 9,403 & 18,724 & 16,305 & 2.340 & 1.104 & 1-12 & 2.154 & 1.621 & 1-16 \\
Fatal injury & 2,462 & 5,060 & 4,439 & 2.474 & 1.327 & 1-11 & 2.492 & 2.067 & 1-19 \\
\midrule
Overall & 178,758 & 385,465 & 309,400 & 2.405 & 0.894 & 1-14 & 1.824 & 1.202 & 1-19 \\
p-value & \textless{}0.001\footnotemark[2] & \textless{}0.001\footnotemark[2] & \textless{}0.001\footnotemark[2] & \textless{}0.001\footnotemark[3] &  &  & \textless{}0.001\footnotemark[3] &  &  \\
\bottomrule
\end{tabular}
\footnotetext[1]{Per each collision: \textit{mean, sd, range(min, max)}}
\footnotetext[2]{Pearson's Chi-Squared test}
\footnotetext[3]{Kruskal-Wallis rank sum test}
\end{table}