\begin{table}[h!]
\caption{Collision severity ordinal classification and OCTraffic dataset counts}
\label{tbl1}
\begin{tabular}{lrrrrrcrrc}
\toprule
\multicolumn{4}{c}{} & \multicolumn{3}{c}{Party Count\footnotemark[1]} & \multicolumn{3}{c}{Victim Count\footnotemark[1]} \\
Severity Level & Crashes & Parties & Victims & \textit{mean} & \textit{std} & \textit{range} & \textit{mean} & \textit{std} & \textit{range} \\
\midrule
Possible injury or pain & 107,448 & 238,719 & 183,249 & 2.437 & 0.839 & 1-12 & 1.677 & 1.032 & 1-16 \\
Minor or visible injury & 56,621 & 116,950 & 101,251 & 2.350 & 0.937 & 1-14 & 2.012 & 1.313 & 1-17 \\
Serious or severe injury & 9,176 & 18,298 & 15,941 & 2.346 & 1.112 & 1-12 & 2.157 & 1.627 & 1-16 \\
Fatal injury & 2,402 & 4,931 & 4,336 & 2.472 & 1.332 & 1-11 & 2.490 & 2.073 & 1-19 \\
\midrule
Overall & 175,647 & 378,898 & 304,777 & 2.406 & 0.894 & 1-14 & 1.825 & 1.204 & 1-19 \\
p-value & \textless{}0.001\footnotemark[2] & \textless{}0.001\footnotemark[2] & \textless{}0.001\footnotemark[2] & \textless{}0.001\footnotemark[3] &  &  & \textless{}0.001\footnotemark[3] &  &  \\
\bottomrule
\end{tabular}
\footnotetext[1]{Per each collision: \textit{mean, sd, range(min, max)}}
\footnotetext[2]{Pearson's Chi-Squared test}
\footnotetext[3]{Kruskal-Wallis rank sum test}
\end{table}